% Akronyme
%--------------------------------------------------------------------------
\newacronym{acn:MS}{MS}{Microsoft}
\newacronym{acn:CD}{CD}{Compact Disc}
\newacronym{acn:AC}{AC}{Aachen}
\newacronym{acn:FH}{FH}{Fachhochschule}
\newacronym{acn:TU}{TU}{Technical University}
\newacronym{acn:UNB}{\"UNB}{Übertragungsnetzbetreiber}
\newacronym{acn:AD}{AD}{Active Directory\protect\glsadd{glos:AD}}

% Glossar
%--------------------------------------------------------------------------
\newglossaryentry{glos:FH}{
	name=FH, 
	description={Fachhochschule}
}
\newglossaryentry{glos:AC}{
	name={AC},
	description={Aachen}
}
\newglossaryentry{glos:AD}{
	name=Active Directory,
	description={Active Directory ist in einem Windows 2000 "Windows
	Server 2003-Netzwerk der Verzeichnisdienst, der die zentrale
	Organisation und Verwaltung aller Netzwerkressourcen erlaubt. Es
	ermöglicht den Benutzern über eine einzige zentrale Anmeldung den
	Zugriff auf alle Ressourcen und den Administratoren die zentral
	organisierte Verwaltung, transparent von der Netzwerktopologie und
	den eingesetzten Netzwerkprotokollen. Das dafür benötigte
	Betriebssystem ist entweder Windows 2000 Server oder
	Windows Server 2003, welches auf dem zentralen
	Domänencontroller installiert wird. Dieser hält alle Daten des
	Active Directory vor, wie z.B. Benutzernamen und
	Kennwörter.}
}
\newglossaryentry{glos:AntwD}{
	name=Antwortdatei,
	description={Informationen zum Installieren einer Anwendung oder des Betriebssystems.}
}

% Zusätzliche Glossare
%--------------------------------------------------------------------------
% Readme
%	newglossary[Metafile_Ending]{}
% 	Reservierte Metafile_Endings:
% 		Main Glossery(glg, gls, glo)
%		Acronyme(alg, acn, acr)
% 	Beispiel:
% 	[abc]{typename}{abc}{abc}{Title}
% 	[slg]{symbolslist}{syi}{syg}{Additional Glossar}
%--------------------------------------------------------------------------
% Definitions
%--------------------------------------------------------------------------
\newglossary[slg]{symbolslist}{syi}{syg}{Symbolverzeichnis}

% Einträge
%--------------------------------------------------------------------------
\newglossaryentry{symb:Pi}{
	name=$\pi$,
	description={Die Kreiszahl.},
	sort=symbolpi,
	type=symbolslist
}
\newglossaryentry{symb:Phi}{
	name=$\varphi$,
	description={Ein beliebiger Winkel.},
	sort=symbolphi,
	type=symbolslist
}
\newglossaryentry{symb:Lambda}{
	name=$\lambda$,
	description={Eine beliebige Zahl, mit der der nachfolgende Ausdruck
		multipliziert wird.},
	sort=symbollambda,
	type=symbolslist
}
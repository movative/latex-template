% Packages
%--------------------------------------------------------------------------
\usepackage[utf8]{inputenc} % UTF8 für Deutsche Umlaute
\usepackage[ngerman]{babel} % Silbentrennung
\usepackage[T1]{fontenc} % 8-Bit Fontencoding
\usepackage{kpfonts} % Kepler-Fonts
\usepackage{setspace} % Zeilenabstand
\usepackage{amsmath, amssymb, stmaryrd} % Math
\usepackage{geometry, chngpage, scrlayer-scrpage} % Seitenränder & Header/Footer
\usepackage{graphicx, svg, tikz} %Abbildungen http://www.tickerverbot.de/grafiken-fur-latex
\usetikzlibrary{patterns,decorations.pathreplacing} % Use tikz
\usepackage{booktabs}% Tabellen 
\usepackage{xcolor, colortbl}
\usepackage{float} \restylefloat{table} % Verschiebt angemessen Abbildungen und Tabellen
\usepackage{scrhack, listings, enumitem} % Schicke Aufzählungen

% Verlinkung von Verzeichnissen, Quellen, WWW-links, etc.
\usepackage
[colorlinks,
pdfpagelabels,
pdfstartview = FitH,
bookmarksopen = true,
bookmarksnumbered = true,
linkcolor = black,
plainpages = false,
hypertexnames = false,
citecolor = black,
filecolor=black,
urlcolor=black]{hyperref}
\usepackage{memhfixc} %hyperref fix
\usepackage{tocloft}
% Verzeichnis
\usepackage[acronym, toc, automake, nonumberlist, nopostdot]{glossaries}
\usepackage[colorinlistoftodos,prependcaption,textsize=tiny]{todonotes} %ToDo Listen
\newcommand{\fehler}[2][1=]{\todo[inline,linecolor=red, backgroundcolor=cyan]{#2}} %https://tex.stackexchange.com/questions/9796/how-to-add-todo-notes\textbf{}
\usepackage{ifthen} % Logik für Einstellungen

% Neue Befehle
%--------------------------------------------------------------------------
% Linienbreite und größe der Abbildungen
\newcommand{\ruleThickness}{2.5pt}
\newcommand{\figurewidth}{0.8\textwidth}
\newcommand{\tablewidth}{1\textwidth}
\newcommand{\spacebetween}[1]{\renewcommand{\arraystretch}{#1}}
\newcommand{\dashrule}{\color{rulecolor}\makebox[\linewidth]{\rule[\dimexpr.5ex-.2pt]{4pt}{.4pt}\xleaders\hbox{\rule{4pt}{0pt}\rule[\dimexpr.5ex-.2pt]{4pt}{.4pt}}\hfill\kern0pt}}
\arrayrulecolor{rulecolor}

% Verzeichniseinstellung
\setlength{\glslistdottedwidth}{.3\linewidth}
\newlistof{equations}{equ}{Formelverzeichnis}
\newcommand{\equations}[1]{\begin{center}#1\end{center}\addcontentsline{equ}{equations}{\protect\numberline{\theequation}#1}\par}

% Nummerierung auf Vakatseiten
\newcommand{\makevakat}{\makeatletter \let\ps@empty\ps@plain\makeatother}
